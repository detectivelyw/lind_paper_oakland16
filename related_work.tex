\section{Related Work}
\label{sec.related_work}
\cappos{extremely weak}

While our metric and the resulting Lind prototype represent a novel approach 
to privileged code protection, it does build on the idea of providing 
strong isolation between the kernel space and the user space 
which has been pioneered by a number of existing techniques. 
In this section, we take a close look at three categories of techniques 
that share similar security goals as Lind.

\subsection{Virtualization}

Programming language virtualization, such as Java \cite{Java}, Lua \cite{Lua}, Silverlight \cite{Silverlight}, 
and JavaScript \cite{JavaScript} are commonly used programming language sandboxes 
that have achieved widespread adoption. The key idea of this technique is 
to run code within a special process that checks the safety of operations. 
These sandboxes combine untrusted application code with an interpreter 
and standard libraries that consolidate routines to perform I/O, network communication, 
and other system sensitive functions. Though many sandboxes implement 
the bulk of their standard libraries in a memory-safe language like Java or C\# \cite{CSharp}, 
flaws in memory-safe code can still pose a threat. In fact, many security critical bugs 
can be found in the standard libraries \cite{JavaBugs}. Researchers have also discovered 
many severe bugs in Java \cite{Java-Lessons}.  Unfortunately, any bug or failure 
in a programming language virtual machine is usually fatal. Lind offers better security 
than the above virtual machines, because of its very small TCB of only about 8K LOC, 
making it is easier to secure. In addition, since the POSIX implementation in Lind is 
written with Repy and contained in a sandbox, bugs and failures in our POSIX implementation 
will be isolated and will not cause fatal damage.

OS virtualization is widely used to isolate untrusted applications that share the same hardware. 
There are a few varieties, such as bare-metal hardware virtualization (Type-I Virtualization), 
which include VMware ESX Server \cite{VMWare-ESX-Server}, XEN \cite{Xen-03}, LXC \cite{LXC}, 
Jails \cite{Jails}, Zones \cite{Zones}, and MS Hyper-V \cite{Microsoft-Hyper-V}. 
Another type is hosted hardware virtualization (Type-II Virtualization), such as VMware Workstation \cite{VMWare-Workstation}, 
VMware Server \cite{VMWare-Server}, MS Virtual PC \cite{Microsoft-Virtual-PC} and VirtualBox \cite{VirtualBox}. 
However, there are limitations with virtualization. 
Virtualization hypervisors suffer from attack vectors, including architectural and software vulnerability, 
and configuration risk. For example, the small code footprint of a hypervisor can become a potential vulnerability 
and lead to serious problems. In addition, OS virtualization systems are made up of a very large body of 
OS code with a complex interface, both of which make it very difficult to guarantee secure isolation. 
Attackers can easily find vulnerabilities to escape these systems and access arbitrary code 
in the underlying host systems. Lind addresses these concerns by using a much smaller and safer TCB.

Library OSes are a useful approach for applications to efficiently obtain the benefits of virtual machines, 
including security isolation, host platform compatibility, and migration. Drawbridge \cite{Drawbridge-11} is a new program 
that uses lightweight processes and a library OS to present a Windows persona to a wide variety of Windows applications. 
This is accomplished by moving a large portion of the OS into the process, and presenting a simplified system 
virtual machine-like interface to each process. This approach brings many of the benefits of VM based temporal, 
spatial and fault isolation properties to a per-process level. Bascule \cite{Bascule}, an architecture for library OS extensions 
based on Drawbridge, allows application behavior to be customized by extensions loaded at runtime. 
Graphene \cite{Graphene-14} is a recent library OS system that seamlessly and efficiently executes both single and 
multi-process applications, generally with low memory and performance overheads. 
It broadens the library OS paradigm to support secure, multi-process APIs, such as copy-on-write fork, signals, 
and System V IPC. Haven \cite{Haven} uses a library OS to implement shielded execution of unmodified server applications 
in an untrusted cloud host. Haven leverages the hardware protection of Intel SGX to defend against 
privileged code and physical attacks, such as memory probes, but also addresses the dual challenges of 
executing unmodified legacy binaries and protecting them from a malicious host. 
The library OS technique is similar to Lind but differs in the fact that existing library OS systems rely heavily on 
the underlying kernels to perform system functions, while Lind only relies on a very limited set of system functions, 
and reconstructs most OS functions with our own safe Repy code. 

\subsection{System Call Interposition}

System call interposition offers a number of properties that make it attractive for building sandboxes, 
though it can be error prone \cite{SCI-04}. Approaches for delegation and filtering have been extensively studied, 
along with their respective tradeoffs between security and performance. 
Janus Version 2 (J2) \cite{Janus0:96, Janus:99} uses filtering and sandboxing, while Ostia \cite{SCI-04} uses a delegation. 
Ostia also provides a hybrid interposition architecture, which allows for kernel level enforcement and user policies. 
The kernel module enforces policy, denies direct access to restricted resources, 
and delegates certain calls to the emulation library, which sends transformed system calls to the agents. 
The agent reads the policy file and handles the delegation of calls. However, system call interposition has many problems. 
OS semantics are very difficult to replicate correctly. Indirect paths to resources are often overlooked, 
and there are side effects to denying system calls \cite{Problems-SCI}. 
Nevertheless, this technique is very useful and has inspired many new techniques, such as library OSes. 
The concepts behind System Call Interposition have evolved into other modern techniques 
and has benefitted many security systems, including Lind. 

\subsection{Software Fault Isolation}
Software Fault Isolation (SFI) is a sandboxing technique in which native instructions can only be executed 
if they do not violate the sandbox's constraints \cite{SFI:93}. Instructions usually exclude any directions that 
could jump outside the program's predefined memory region or execute a system call. 
Programs must use a specific compiler, which enforces these constraints before the programs are validated on load. 
Nooks \cite{Nooks:03} isolates Linux kernel modules, so that if they crash, the whole kernel does not crash as well. 
They use two techniques: sandboxing and micro-reboots. Nooks uses both as SFI-based and user-level sandboxes. 
Micros-reboots conceptually refresh small parts of the system. 
Google's Native Client \cite{NaCl-09} adopts SFI to execute untrusted x86 binary code safely. 
We leveraged this technique and used NaCl in our system Lind. The difference is that SFI doesn't handle the manner 
in which systems interact with the underlying OS. In NaCl, to perform system calls, 
it relies on a trusted service runtime for thread creation, memory management, and other system services. 
In Lind, we redirect system call requests to our safe POSIX interface, 
which allows Lind to control its own interaction with the underlying OS kernel.
