\section{Introduction}
\label{sec.introduction}

To run multiple applications on a computer, it is critical to securely
manage access to the underlying hardware. In modern computer systems,
either a hypervisor - a virtual machine monitor (VMM), or an 
operating system (OS) kernel performs this important function. Unfortunately, code within an OS kernel may contain flaws and vulnerabilities. If the OS kernel is attacked by a malicious adversary, potential vulnerabilities can enable the attacker to have unrestricted access to the system.


%One critical flaw, discovered in the Linux kernel in the \texttt{futex} subsystem call can allow an attacker to gain ring 0 control via the \texttt{futex} syscall, and potentially execute arbitrary code with kernel mode privileges~\cite{CVE-2014-3153}. \cappos{Possibly omit this sentence.}


There is a diverse set of defensive technologies to protect kernels from 
attacks, including OS virtualization (e.g., Xen, KVM, VirtualBox), system call filtering \cite{Janus:99},\cite{SCI-04}, and library OSes \cite{Bascule},\cite{Drawbridge-11}. Common security wisdom is that by running software in a virtual machine, one can prevent the attacker from exploiting flaws in the underlying kernel. However, the security of virtualization itself is a  challenging issue \cite{Tal}. As we will demonstrate in this paper, despite employing virtualization techniques, still one third of vulnerabilities on the Linux kernel are not prevented from being exploited. This rate highlights substantial risk to run applications in virtualized envinronments.

In this paper, we develop a metric that helps to identify where
within the kernel vulnerabilities are likely to be located. For this purpose, we
examined 40 kernel patches that fix severe Linux kernel security bugs
and analyzed the lines of the kernel where those bugs occurred. The results of this analysis shows that kernel paths used by popular applications contain fewer
security bugs, and therefore, can be exposed with less risk. 

While limiting access to kernel code is important, it is
insufficient to build a secure virtualization system.  First, if a complex
program is prevented from accessing part of the kernel, its functionality 
must exist somewhere else for this program to work.  Second, it is 
typical for virtualization systems to add new privileged code.
As a result, a vulnerability in the privileged codebase is as much of a security 
risk as a flaw in the kernel.  %However, this privileged code needs to be
%complex because the system must implement complex 
% must exist somewhere or else applications will not run. 
For example, a vulnerability in VMWare's codebase caused by buffer overflows in the VIX
API could allow local users to escape out of the guest VM and 
gain privilege escalation to execute arbitrary code in the host
OS, even shellcode to access the kernel of the host OS~\cite{CVE-2008-2100}.  
%\cappos{Revise this paragraph further}

CVE-2011-1751 was another bug due to missing check in KVM's QEMU emulation of PCI-ISA bridge. This enables an attacker to utilize other techniques \cite{Virtunoid} for root exploit in the host OS being triggered from a guest.

%To address this issue, 
To this end, we propose a new design for virtualization 
systems called ``safely-reimplement'' that restricts access to only the
kernel paths that are mostly used by common applications.  To restrict the privileged
code, we first build a minimal, sandboxed environment that also only uses 
common kernel paths.
We then implement a POSIX interface inside of this safe sandbox. For complex and dangerous system functions, which may access the risky portion of the kernel, 
input to POSIX is reimplemented by our own code within a sandbox. Any bugs or failures within the implementation of those complex system functions 
will be contained by the sandbox, and will not have the chance to reach 
and trigger risky portions of the kernel. Because of this added security, we dub this as a "safe-reimplementation" design. \yanyan{the sandbox only has
safe code. how can unsafe code to be trapped "within" the sandbox?} Therefore, even if the code
is vulnerable to an attack, it is unable to trigger
kernel paths that are less commonly used or tested.

This implementation helped us to understand which kernel paths are hardest to
safely-reimplement and provided key insights about what kernel paths system
designers should give the highest degree of scrutiny. In turn, we then use this design paradigm to develop a virtualization system called
Lind.  Lind uses Google Native client for software fault isolation (memory
safety of the application) and the Repy sandbox~\cite{Repy-10} to contain the POSIX
implementation and to provide access to the kernel.   \yanyan{shouldn't it
be the other way round? first fully understand which paths are hard, and
then use this insight to design Lind.}

To evaluate the effectiveness of Lind, we first captured the 
kernel traces from user programs run in Lind and four other virtualization
systems.
%and compare their kernel traces. 
We then examined historical
kernel bug reports to verify which trace was more likely to trigger bugs.
Results showed that applications run in Lind were the least likely to
trigger kernel bugs: only one triggered out of the 35 kernel vulnerabilities 
(2.9\%). In contrast, the virtualization systems built
without our metric triggered more vulnerabilities (23-40\%). This
suggests that our metric can help effectively design and build more secure
virtualization systems.

The main contributions of this paper are as follows: %\cappos{revise}
\begin{itemize}
\item Proposing a novel metric for quantitatively measuring and evaluating the
security of privileged code, such as in an OS kernel. The proposed metric examines
the safety of the kernel trace, at the lines-of-code level, generated by
running user applications. This metric is helpful for the design of secure systems.
%\yanyan{what is "generated by producing design recommendations"?}
\item Validation of the key hypothesis that commonly used kernel paths contain fewer bugs according to our metric.
 
\item Building a secure ``safely-reimplement'' design by examining our key hypothesis and proposed metric. 
%that commonly used kernel paths contain fewer bugs. 

The key idea in the proposed design is reimplementation of risky system calls inside a
sandbox that only uses safe kernel paths. 
\item With this new design, we implemented a sandbox security system, Lind, that
provides a secure environment for applications and strong protection for
the kernel.
\item Results show that the implementation of Lind only triggers one (2.9\%) of
the kernel vulnerabilities we examined, while systems built without using
our metric trigger about an order of magnitude more vulnerabilities. This
suggests that our metric can help design and build virtualization systems
with greater security. 
\end{itemize}
\cappos{Quantitative analysis for sandboxes}


The remainder of this paper is organized as follows. 
Section \ref{sec.motivation-and-background} describes the motivation and background material. Section \ref{sec.metric}, introduces the proposed coverage safety metric. Using this metric, the 
``safely-reimplement'' design pattern has been discussed in Section \ref{sec.design}. Section \ref{sec.implementation}, explains Lind: the sandbox security system. Section \ref{sec.evaluation} applies the quantitative  measures for comparing the security and efficiency of Lind with other virtualization systems. \ref{sec.limitation} outlines the limitations and future work. Section \ref{sec.related_work} reviews the related work on security through several isolation mechanisms. Finally in Section \ref{sec.conclusion}, we present our concluding remarks.

