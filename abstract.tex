\begin{abstract}

Building secure systems that allow untrusted programs to run without triggering vulnerabilities 
in the underlying privileged code of an operating system (OS) kernel or hypervisor is very challenging. 
Despite substantial effort by security researchers and system developers to eliminate these flaws, 
exploitable vulnerabilities can still be found. Techniques to protect against intentional or 
unintentional triggering of these vulnerabilities, such as system call filtering, 
operating system virtualization, and library OSes have had limited success, 
but often harbor their own exploitable vulnerabilities. As a result, they can not fully prevent 
buggy programs from triggering flaws, or attackers from leveraging these bugs for their own purposes.

In this paper, we introduce a novel security design approach that leverages controlled kernel access 
to protect privileged code from exploitation by untrusted programs. We start by analyzing the reasons 
why existing techniques have been less than effective. We then present a new metric 
to determine where kernel flaws are likely to be located, based on the hypothesis that 
commonly-used kernel paths, executed by programs used on a daily basis, 
contain fewer vulnerabilities than less-used paths. Next, we discuss how this insight 
was used to devise a novel design pattern that reimplements essential OS functionality inside a sandbox. 
This pattern has a very small trusted computing base (TCB) that only accesses commonly used kernel paths. 
Lastly, we show how this design pattern was used to implement  a prototype security system called Lind. 
Our experimental results show that programs run inside Lind triggered about 10x fewer kernel vulnerabilities 
compared to existing virtualization systems like VirtualBox, VMWare Workstation, Docker and Graphene.  

\end{abstract}