\begin{abstract}

Building secure systems that allow untrusted programs to run without
triggering vulnerabilities in the underlying privileged code of an
operating system (OS) kernel or hypervisor is very challenging. \yanyan{sentence too long}Despite
substantial effort by security researchers and system developers to
eliminate these flaws, exploitable vulnerabilities can still be found. \gholami{maybe rephrase to: ... these flaws, still it is long away to make existing OS kernels secure.}
Techniques to protect against intentional or unintentional triggering of
these vulnerabilities, such as system call filtering, operating system
virtualization, and library OSes have had limited success.  The problems
of these techniques are two-fold.  First, these systems add trusted code that often has new
exploitable vulnerabilities. Second, the portions of the underlying kernel
that these systems use may enable the attacker to exploit vulnerabilities.
As a result, they can not fully prevent buggy programs from triggering
flaws, or attackers from leveraging these bugs. 
%for their own purposes.

\cappos{Rework...}
In this paper, we introduce a novel security design that leverages
controlled kernel access to protect privileged code from exploitation by
untrusted programs. \gholami {Maybe to say it's called Lind instead of mentioning it at the end of this paragraph} We start by analyzing the effectiveness of existing
solutions and explore the reasons why existing techniques are not
effective. \gholami {I think this sentence is good inside the paper than abstract?} We then present a new metric to determine where kernel flaws are
likely to be located, based on a hypothesis that commonly-used kernel
paths, executed by applications on a daily basis, contain fewer
vulnerabilities than less-used paths. Next, we use this insight to devise a
novel design that reimplements \yanyan{replace reimplement with 
something else?} essential OS functionality inside a
sandbox that only accesses commonly used kernel paths \gholami {is this rephrase of the first sentence?}. Lastly, we use this design to prototype a security system called Lind.\gholami {I'm little bit confused here..} Our experiment results show that \gholami{Lind confines adversarial attacks through executing malicious code with less than... ?} an attacker executing code inside Lind can trigger less than 3\% of zero day kernel attacks, which is about an order of
magnitude less than existing systems like VirtualBox (40\%), VMWare Workstation (31\%), Docker (23\%), and Graphene (23\%).


\end{abstract}
