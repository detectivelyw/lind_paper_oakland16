\begin{abstract}

Building secure systems that allow untrusted programs to run without
triggering vulnerabilities in the underlying privileged code of an
operating system (OS) kernel or hypervisor is very challenging. Despite
substantial effort by security researchers and system developers to
eliminate these flaws, exploitable vulnerabilities can still be found.
Techniques to protect against intentional or unintentional triggering of
these vulnerabilities, such as system call filtering, operating system
virtualization, and library OSes have had limited success.  However, there
are two problems.  First, these systems add trusted code that often has new
exploitable vulnerabilities. Second, the portions of the underlying kernel
that these systems use may enable the attacker to exploit vulnerabilities.
As a result, they can not fully prevent buggy programs from triggering
flaws, or attackers from leveraging these bugs for their own purposes.

\cappos{Rework...}
In this paper, we introduce a novel security design approach that leverages
controlled kernel access to protect privileged code from exploitation by
untrusted programs. We start by analyzing the effectiveness of existing
solutions and explore the reasons why existing techniques are not
effective. We then present a new metric to determine where kernel flaws are
likely to be located, based on the hypothesis that commonly-used kernel
paths, executed by programs used on a daily basis, contain fewer
vulnerabilities than less-used paths. Next, use this insight to devise a
novel design pattern that reimplements essential OS functionality inside a
sandbox that only accesses commonly used kernel paths. Lastly, we use this
design pattern to implement  a prototype security system called Lind. Our
experimental results show an attacker executing code inside Lind can 
trigger less than 3\% of zero day kernel attacks, which is about an order of
magnitude less than existing systems like VirtualBox (40\%), VMWare
Workstation (31\%), Docker (23\%), and Graphene (23\%).  


\end{abstract}
