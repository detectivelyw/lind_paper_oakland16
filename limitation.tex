\section{Study Limitations/Exceptions and Future Work}
\label{sec.limitation}

While these early results are promising, 
it should be noted that there were a few limitations on the scope 
of the study that may need to be addressed in future studies. 
First of all, our metric relies on identifying lines of code in the kernel that 
are executed when running applications in security systems. 
This would be difficult to utilize in systems that modify the kernel in substantial ways. 
The consequences could be inaccurate, or incomplete results. 

A second issue is that our criteria for determining if a kernel trace is safe or risky 
was to check it against a list of historical kernel bugs. However, not all bugs can be 
accurately checked in this manner. For example, some bugs caused by a race condition, 
cannot be identified by directly checking if certain lines of code are executed. 
For complicated bugs that involve defects in the internal kernel data structures, 
or require complex triggering conditions across multiple kernel paths, our metric 
will not be able to accurately determine whether or not those bugs have been triggered. 
In such cases, more complex metrics might be needed for detection.  

Lastly, while Lind executes programs like Apache and Tor, 
it does not support every system call or every possible set of arguments. 
For example,  symbolic links are not supported.  While the SafePOSIX implementation 
could be extended to do so, we leave this for future work. 
Also, as mentioned in the previous section, Lind has not been optimized for performance, 
so the results we present here should be taken as a baseline of what is possible.

There were also some avenues we intentionally excluded in this initial study that could form 
the basis for interesting research projects in the future. First, we chose not to explore bugs 
within the applications themselves.  We also did not test if our metric applied to hypervisors, 
which would certainly be an important area to address. 

Other future investigations could center on the use of our metric in other popular operating systems, 
such as Windows and Mac OS. Our experiments were limited to Linux kernel 3.14.1 
and some of the typical virtualization systems that existed in Linux. 
It would be interesting and potentially quite beneficial to the advance of secure systems 
if similar tests could be run in those other widely-used operating systems. 