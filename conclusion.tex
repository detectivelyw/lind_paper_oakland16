\section{Conclusion}
\label{sec.conclusion}

Though the idea of isolating untrusted user applications from the underlying privileged code 
to avoid its exploitation by bugs has been acknowledged through many different implementations, 
there is no standard method for creating this isolation. 
And, isolation by itself does not guarantee the security of the system.

In this paper, we proposed a new metric that quantitatively measures and 
evaluates execution of the kernel code when running user applications. 
We stated and tested our key hypothesis that commonly used kernel paths contain fewer bugs. 
Using our metric, we generated findings that suggest the hypothesis is reasonable and solid.

Based on the findings from our metric, a new design for building secure systems was created. 
We implemented a sandbox system called Lind, which securely reconstructs complex, 
yet essential OS functionality inside a sandbox. The sandbox itself is designed to 
have a minimized trusted computing base (TCB) and only interact with the kernel in a minimal and safe way. 

Evaluation results have shown that Lind is least likely to trigger historically-reported kernel bugs, 
when compared to other virtualization systems, such as VirtualBox, VMWare Workstation, Docker, and Graphene. 
This suggests that systems using our design
are likely to be more secure.

We make all of the data and source code for this paper available to other 
researchers.  Lind is open source and available at: XXX.  The kernel 
trace data is freely available on the Lind website.  For access
to the the kernel exploit code created in this study, please contact the 
authors.


